\section{Conclusion}
Due to urbanization, various aspects of people's lives, such as traffic flow, power consumption, air quality, and social network, change rapidly. In recent years, policy-makers and scientists have realized the urgent need for a transition to an environmentally friendly and sustainable city, and have proposed \emph{urban computing} research. One of the critical components of intelligent cities is smart building research. In particular, energy distribution and consumption are important topics because of their two-fold impact on society. First, power is an essential resource which has a great impact on our economy, so effective consumption is important. Secondly, the large impact the energy industry has on the environment, such as climate change, makes it an extremely sensitive and relevant topic of research. 

The central theme of this paper is using data analytics methods to tackle major issues in smart building. We provide a general framework and propose analytical solutions within that framework. The data mining techniques we use for discovering knowledge from the datasets are temporal mining and probabilistic models. We classify smart building events into two categories: device-related profiles and human-related events. This paper provides strategies to handle these two challenging energy-related problems. 

\begin{itemize}
\item \textbf{Characterization of electrical devices by energy disaggregation}: We studied energy consumption inside a building with an non-intrusive approach,  extracted the hidden patterns, and disclosed the correlation among these patterns. Furthermore, we connected these patterns with devices to infer the electricity usage of individual devices. 
We demonstrate that with the use of frequent-episode mining in conjunction with temporal mining techniques, we can effectively glean insights into usage patterns of electrical devices. Our approach describes a novel motif discovery approach that utilizes on/off events to unravel the operation frequency and duration of devices. We show that the our approach is very adroit at discerning multiple power levels and at effectively untwining the combinatorial operation of the devices. Moreover, we also show that this approach is not just an aid to disaggregation but, as a byproduct, also extracts temporal episodic relationships that shed insight into consumption patterns.

%\item \textbf{Characterization of water use ends by water disaggregation}: We characterized the water usage patterns inside a building, and inferred underlying water use ends. Additionally, we associated these patterns according to time to derive the activities of the people inside. 
%To improve upon our initial work, we propose a semi-supervised recursive multivariate piecewise motif mining approach for both energy and water disaggregation. Since the algorithm operates in two phases and effectively filters out appliances that have a large power consumption in the first stage, it can effectively discover usage patterns of smaller appliances. This insight provided by our approach allows for more precise energy disaggregation. Moreover, this approach can be effectively utilized to identify continuously variable loads, like outdoor heating. Similarly, by regarding hot and cold water as two phases, we can separate many water use ends.  

\item \textbf{Characterization of activities of daily life by occupancy prediction}: We investigated the activities patterns of people inside a residential building. Then we connected these patterns with a graphical model. Using the graphical model, we inferred whether the room/house was occupied. 
We demonstrate that by integrating the mixture EGH model and 
kNN together as a hybrid approach, we get the best prediction result. In our work, we formulate the problem as one of temporal mining; the activities inside the building are abstracted as episodes, and each episode is connected with an episode-generative HMM model. We mine the activity patterns according to the time and gap; both the duration of each type of activity, and the gap between two consecutive events are limited to be within a proper range. 
\end{itemize}

This research demonstrates that data analysis methods are able to solve smart building challenges. The proposed heuristic approaches utilize the characteristics of two typical profiles in smart buildings, thereby fundamentally paving the way to advanced controlling and scheduling systems for devices inside the buildings. 

%========
%In this day and age of big data and big compute, data mining is an important tool  the world. Technologies like Internet of Things and the availability of data such as traffic flow, power consumption, air quality, social networks, etc. have made \emph{urban computing} a burgeoning research topic. One of the critical components of \emph{urban computing} research is the analysis of smart buildings. Particularly, energy disaggregation and distribution are important topics because of its two-fold impact on society. Firstly it is an essential resource, which has a great impact on our economy so effective distribution is important. Secondly, the larger impact the energy industry has on the environment, like climate change, makes it an extremely sensitive and relevant topic of research. Thus, in this work, we focus on two important energy related problems in a smart building setting viz. resource disaggregation and occupancy prediction. 

\textbf{Future Work}

This paper has opened up many opportunities for future work. From a theoretical perspective, 
the probabilistic model of piecewise events has great research potential. We can try to connect frequent piecewise episodes with a generalized probabilistic model, as in \cite{laxman2005discovering}. %Furthermore, we can associate the dynamic time-warping models used in water disaggregation with similar probabilistic models. 

%In the aspect of research topics, the possible future works are stated as following. 
While we have made significant headway in energy disaggregation, there is significant room for improvement. One of the immediate extensions is to incorporate more features (in the multi-phase aggregate) when a device turns on. The sudden spikes in the aggregated data, when normalized, can indicate the \emph{startup shape}, which can be correlated to a device, and hence improve the overall effectiveness of device prediction, thus leading to more accurate energy disaggregation. Moreover, we can significantly improve the temporal mining approach to disaggregate more devices. Furthermore, our disaggregation algorithms can be explored for water disaggregation as well. We will exploit our temporal mining algorithms, integrated with dynamic time warping and motif mining, to propose an algorithm to effectively conduct home-level disaggregation. We can also extend our disaggregation algorithms for bandwidth distribution for internet service providers. Using our disaggregation algorithms, we can decipher the device-level internet usage and plan for effective distribution to a home/neighborhood. 

Solving energy disaggregation is an important practical problem, which, when resolved, results in a highly monetizable insight: consumption patterns. As a result, policy makers and energy distributors can design packages for consumers based on their needs. This will enable both consumers and distributors to effectively and \emph{smartly} buy and sell power that is highly customized to the needs of a home.


The occupancy prediction work lends itself to future extensions via hybrid approaches. We can integrate kNN and a mixture of EGH, which has the best performance on the sensor data set. One of the future directions is to incorporate GPS-based information to track movements of the house residents. This has the potential to be an excellent surrogate to automated power control of devices in a home. Another interesting problem to tackle is holiday occupancy prediction. The occupancy patterns for these days are completely different. For example, on certain weekdays, a person may never go out. Therefore the occupancy prediction probably depend more on date than on other indoor activities. 

One of the critical problems to ensure that we can reduce high energy consumption by automatically controlling the temperature regulation systems. Moreover, with the advent of more mainstream technologies like \emph{Nest} and other intelligent automatic control systems that remotely control the thermostat, occupancy prediction is a crucial parameter in determining the settings. Furthermore, comfort levels based on individuals can determine the temperature at which different parts of the residence should be set. Thus optimizing energy usage and maximizing user comfort are very important and immediate problems to solve.

Research in the domain of energy consumption has a two-fold impact on the society we live in. The rapid urbanization of our society requires the \emph{smart} and optimized distribution of power to meet the demands of the city. Logistical problems in power distribution, particularly meeting the high-volume requirement with minimal failure, are important problems. Moreover, the larger problem here is to ensure that we leave a small carbon foot print. Even though renewable energy resources are being tapped, fossil fuels are still the fundamental source of energy in our society. The optimized distribution of power and a well-established balance between supply and demand can ensure that we use only just as much of the fossil fuels as we need. As responsible researchers, it is important that we turn our talents towards this overarching goal of saving our planet and doing every thing we can to ensure that it is in good shape when we hand it over to our next generation.
