\section{Evaluation}
%The evaluation tools has been discussed in prior work \cite{liang2010load}.
%
%To evaluate whether each devices is disaggregated or not,
%we need to evaluate two parts, one is how much energy
%is evaluated compared to the ground truth. Another is
%among those disaggregated energy, how much percentage
%are disaggregated right.
%
We use precision, recall and F-measure in our evaluation. The standard
definitions of these metrics are:
%\begin{equation}
$\textrm{precision}=\frac{TP}{TP+FP}$, 
%\end{equation}
%\begin{equation}
$\textrm{recall}=\frac{TP}{TP+FN}$,
%\end{equation}
%\begin{equation}
$\textrm{F-measure}=\frac{1}{\frac{1}{\textrm{precision}}+\frac{1}{\textrm{recall}}}$
%\end{equation}

We need to define the notions of true/false positives
and negatives in the context of disaggregation.

Let us suppose there is a ground truth time series $x$ with length T, 
and denote the corresponding disaggregated time series by $\hat{x}$.
For any time $t \in (0, T)$, there are two values: the
ground truth value of device $m$ is $x_t^{(m)}$ and the disaggregated value
$\hat{x}_t^{(m)}$. We define a parameter $\rho$ for the range of
true values $x_t^{(m)}$, and another parameter $\theta$
as the noise.
For any given measurement, 
there are four total power values or water usage values of device $m$ at
each point: true positive $TP^{(m)}$,  false negative $FN^{(m)}$,
true negative $TN^{(m)}$, and false positive $FP^{(m)}$.

\noindent
1. When $x_t^{(m)} > \theta$ and  $\hat{x}_t^{(m)}> \theta  $,
at this point the disaggregation is a true positive.
There are three situations in turn:

\noindent
1.1. When $  x_t^{(m)} \times (1-\rho) <  \hat{x}_t^{(m)} <  x_t^{(m)} \times (1+\rho)  $, then
\begin{eqnarray*}
 TP^{(m)} &=& \hat{x}_t^{(m)} \\
 FN^{(m)}&=&FP^{(m)} = TN^{(m)}=0
\end{eqnarray*}

\noindent
1.2. When $ \hat{x}_t^{(m)} < x_t^{(m)} \times (1-\rho)$ , then only
the disaggregated power or water usage is considered as true positive and
the power or water usage that is not disaggregated is regarded as a false negative:
\begin{eqnarray*}
TP^{(m)}&=&\hat{x}_t^{(m)} \\
FN^{(m)}&=&x_t^{(m)} - \hat{x}_t^{(m)} \\
FP^{(m)}&=&TN^{(m)}=0
\end{eqnarray*}
1.3 When $ \hat{x}_t^{(m)}>  x_t^{(m)} \times (1+\rho) $, then
the disaggregated power or water usage is a true positive, and those values
which are greater than the truth values are treated as false positive.
\begin{eqnarray*}
TP^{(m)}&=&\hat{x}_t^{(m)} \\
FP^{(m)}&=&\hat{x}_t^{(m)} - x_t^{(m)} \\
FN^{(m)}&=&TN^{(m)}=0
\end{eqnarray*}
2. When $x_t^{(m)} > \theta$ and  $\hat{x}_t^{(m)}< \theta  $,
at this point the disaggregation is a false positive.  Then,
\begin{eqnarray*}
FP^{(m)}&=&x_t^{(m)} \\
TP^{(m)}&=&FN^{(m)} =TN^{(m)}=0
\end{eqnarray*}
3. When $x_t^{(m)} < \theta$ and  $\hat{x}_t^{(m)} > \theta  $,
at this point the disaggregation is a false negative. Then,
\begin{eqnarray*}
FN^{(m)}&=&x_t^{(m)} \\
TP^{(m)}&=&FP^{(m)} = TN^{(m)}=0
\end{eqnarray*}
4. When $x_t^{(m)} < \theta$ and  $\hat{x}_t^{(m)} < \theta  $,
at this point the disaggregation is a true negative.  Then,
\begin{eqnarray*}
TP^{(m)}&=&FN^{(m)} = FP^{(m)} = TN^{(m)}=0
\end{eqnarray*}
%In practice, we use different
%$\rho$ and $\theta$ values depending on the
%dataset. For instance, considering the datasets described below,

In our experimental dataset, we set
$\theta=100$ and $\rho=0.2$. 
Although the maximal power consumption of all these devices is 11000W, 
we can still set $\theta < 11000 * 0.1$ because we apply multivariate piecewise motif mining recursively, 
so the devices which consume a large amount of power are deleted in the first few rounds. 
Therefore the power noise which is caused by the high-power electronic devices 
is greatly decreased. 
